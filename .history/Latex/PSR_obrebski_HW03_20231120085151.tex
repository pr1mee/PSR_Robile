
\documentclass[11pt,a4paper]{article}

%\usepackage{graphicx}
\usepackage{amsmath}
 \usepackage[normalem]{ulem}
 \useunder{\uline}{\ul}{}
 \usepackage{lscape}
 \usepackage[table,xcdraw]{xcolor}
\title{Selection and implementation of technical approaches for path planning using ROS2 Humble}
\author{Miguel Obrebski}
\date{} % Use \date{} to display no date

\begin{document}

\maketitle

% Your content goes here

\section{Task description}
\begin{enumerate}
    \item Skim through the papers and other online resources that you collected and select two promising approaches for your topic.
    \item Explain your reasons for selecting these approaches.

          For this you have to imagine the scenario in which the robot is moving through the university campus. Think about how the data would be in this scenario (e.g.\
          images of people, speech, language, lighting, background noise, motion blur, etc.). Would this affect your choice of the approaches?
    \item Identify the hardware and software requirements for the selected approaches.
    \item Add the answers for all the above items in a document and upload in PDF format.
    \item Start implementing the selected approaches.
    \item Set up a GitHub/GitLab repository for your code and mention the link in the PDF document in Item 4.
    \item During the class on 17.11., present the progress in technical implementation:

          \begin{enumerate}
              \item Programming language, libraries, dependencies
              \item Technical issues you are facing
              \item Development plan until 15.12.2023
          \end{enumerate}
    \item If you cannot attend the class on 17.11., include your answer for Item 7 in the PDF document.
\end{enumerate}
Note: The sections are not ordered to fit the tasks' order.


\section{Selecting possible approaches}
\subsection{Considerations}
\subsubsection[short]{General}
\begin{itemize}
    \item Number of nodes?
    \item Number of dimensions? 2 or 3?\\
          2 to make it easier.\\
          Different floors? put them next to each other and connect the elevators
\end{itemize}

\subsubsection{General Ideas}
\begin{itemize}
    \item Dynamic Programming or Cache?
\end{itemize}
\subsubsection{Algorithm-specific}
\begin{itemize}
    \item Runtime
    \item Memory/Storage usage
\end{itemize}


% Please add the following required packages to your document preamble:
% \usepackage[table,xcdraw]{xcolor}
% Beamer presentation requires \usepackage{colortbl} instead of \usepackage[table,xcdraw]{xcolor}
% \usepackage[normalem]{ulem}
% \useunder{\uline}{\ul}{}
% \usepackage{lscape}
\begin{landscape}
    \begin{table}[]
        \centering
        \begin{tabular}{r|ccc|ccc|l|l}
            \textbf{Algorithm} & \multicolumn{3}{c|}{\textbf{Runtime}}           & \multicolumn{3}{c|}{\textbf{Memory}}            & \textbf{Notes}           & {\ul \textbf{Acceptable Solution?}}                                                                                                                                                                        \\ \hline
                               & \multicolumn{1}{c|}{min}                        & \multicolumn{1}{c|}{med}                        & max                      & min                                 & med & max          &                                                                                                                &                                \\ \hline
            Dijkstra           & \multicolumn{1}{c|}{->}                         & \multicolumn{1}{c|}{->}                         & \(O(n log(n)+m)\)        & ->                                  & ->  & \(O(n + m)\) & with fibonacci heap                                                                                            & optimal                        \\
            A*                 & \multicolumn{1}{c|}{}                           & \multicolumn{1}{c|}{}                           & \(O(n^2)\) or \(O(b^d)\) & \(O(n)\)                            &     & \(O(b^d)\)   & \begin{tabular}[c]{@{}l@{}}heavily based\\ on heuristics;\\ memory should\\ be the bottleneck (?)\end{tabular} & optimal                        \\
            IDA*               & \multicolumn{1}{c|}{}                           & \multicolumn{1}{c|}{}                           & \(O(b^d)\)               & \(O(n)\)                            &     & \(O(d)\)     & takes years (?)                                                                                                & optimal, but too long          \\
            Best-First-Search  & \multicolumn{1}{c|}{}                           & \multicolumn{1}{c|}{}                           & \(O(b^d)\)               & \(O(n)\)                            &     &              &                                                                                                                & not optimal                    \\
            \rowcolor[HTML]{96FFFB}
            Floyd-Warshall     & \multicolumn{1}{c|}{\cellcolor[HTML]{96FFFB}->} & \multicolumn{1}{c|}{\cellcolor[HTML]{96FFFB}->} & \(O(n^3)\)               & ->                                  & ->  & \(O(n^2)  \) & \begin{tabular}[c]{@{}l@{}}not efficient for\\ big graphs;\\ this is from all to all\end{tabular}              & optimal, very interesting      \\
            RRT                & \multicolumn{1}{c|}{->}                         & \multicolumn{1}{c|}{->}                         & \(O(K \log K)\)          & ->                                  & ->  & \(O(K)    \) & near optimal                                                                                                   & near optimal, very interesting \\
            \rowcolor[HTML]{96FFFB}
            RRT* (or similar)  & \multicolumn{1}{c|}{\cellcolor[HTML]{96FFFB}->} & \multicolumn{1}{c|}{\cellcolor[HTML]{96FFFB}->} & \(O(K \log K)\)          & ->                                  & ->  & \(O(K)    \) &                                                                                                                & near optimal, very interesting \\
            Jump Point Search  & \multicolumn{1}{c|}{}                           & \multicolumn{1}{c|}{}                           & \(O(n log n)\)           & ->                                  & ->  & \(O(n)    \) &                                                                                                                & optimal for grid-maps          \\
            Trace              & \multicolumn{1}{c|}{}                           & \multicolumn{1}{c|}{}                           & \(O(V + E)\)             & \(O(n)         \)                   &     & \(O(V + E)\) & \begin{tabular}[c]{@{}l@{}}not optimal,\\ weird movement\end{tabular}                                          & no
        \end{tabular}
    \end{table}
    The blue are selected
\end{landscape}


% Fast find the target
% Move
% Continue moving to direction while checking for obstacles
% \begin{enumerate}
%     \item Best Algorithms
%     \item Dynamic or not dynamic?\newline
%           What to pay attention about:\newline
%           Storage, Processing power\newline
%           If not much storage:\newline
%           Caching or just calculating.\newline
%           Questions: How much Storage needed? How much time needed?\newline
%           Algorithms: A* writes the shortest paths from a to b in storage, uses this as new route or something, or Floyd and Warshall algorithm
% \end{enumerate}


\section{Reasons for selection of approaches}
\subsection{Floyd-Warshall algorithm}
Seems like a good solution to preplan all paths instead of calculating manually.\\
Should be resource-efficient.\\
FWA was always interesting for me.
\subsection{RRT* algorithm(s)}
Seems to be running edge in motion planning.\\
Interesting approach. \\
It's something else.
\subsection{Scenarios}
Sorry, for now I just wanna come from A to B.
\section{Identify Hardware and Software Requirements}
For now see above
\section{Start implementing}

\section{Setup GitHub/GitLab repo}

\section{Present progress in technical solution}

\subsection{Programming language, libraries, dependencies}
\subsection{Technical issues you are facing}
\begin {enumerate}
\item Extreme problems getting ROS and Gazole to work, maybe consider making a clean aio tutorial?
\end{enumerate}
\subsection{Development plan until 15.12.2023}

\section{Conclusion}
Conclude your homework here.
% Bibliography (if needed)
% \bibliographystyle{plain}
% \bibliography{your_bibliography_file}


\end{document}
